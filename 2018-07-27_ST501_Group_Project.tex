\documentclass[]{article}
\usepackage{lmodern}
\usepackage{amssymb,amsmath}
\usepackage{ifxetex,ifluatex}
\usepackage{fixltx2e} % provides \textsubscript
\ifnum 0\ifxetex 1\fi\ifluatex 1\fi=0 % if pdftex
  \usepackage[T1]{fontenc}
  \usepackage[utf8]{inputenc}
\else % if luatex or xelatex
  \ifxetex
    \usepackage{mathspec}
  \else
    \usepackage{fontspec}
  \fi
  \defaultfontfeatures{Ligatures=TeX,Scale=MatchLowercase}
\fi
% use upquote if available, for straight quotes in verbatim environments
\IfFileExists{upquote.sty}{\usepackage{upquote}}{}
% use microtype if available
\IfFileExists{microtype.sty}{%
\usepackage{microtype}
\UseMicrotypeSet[protrusion]{basicmath} % disable protrusion for tt fonts
}{}
\usepackage[margin=1in]{geometry}
\usepackage{hyperref}
\hypersetup{unicode=true,
            pdftitle={ST501 Group Project},
            pdfauthor={Our Group},
            pdfborder={0 0 0},
            breaklinks=true}
\urlstyle{same}  % don't use monospace font for urls
\usepackage{color}
\usepackage{fancyvrb}
\newcommand{\VerbBar}{|}
\newcommand{\VERB}{\Verb[commandchars=\\\{\}]}
\DefineVerbatimEnvironment{Highlighting}{Verbatim}{commandchars=\\\{\}}
% Add ',fontsize=\small' for more characters per line
\usepackage{framed}
\definecolor{shadecolor}{RGB}{248,248,248}
\newenvironment{Shaded}{\begin{snugshade}}{\end{snugshade}}
\newcommand{\AlertTok}[1]{\textcolor[rgb]{0.94,0.16,0.16}{#1}}
\newcommand{\AnnotationTok}[1]{\textcolor[rgb]{0.56,0.35,0.01}{\textbf{\textit{#1}}}}
\newcommand{\AttributeTok}[1]{\textcolor[rgb]{0.77,0.63,0.00}{#1}}
\newcommand{\BaseNTok}[1]{\textcolor[rgb]{0.00,0.00,0.81}{#1}}
\newcommand{\BuiltInTok}[1]{#1}
\newcommand{\CharTok}[1]{\textcolor[rgb]{0.31,0.60,0.02}{#1}}
\newcommand{\CommentTok}[1]{\textcolor[rgb]{0.56,0.35,0.01}{\textit{#1}}}
\newcommand{\CommentVarTok}[1]{\textcolor[rgb]{0.56,0.35,0.01}{\textbf{\textit{#1}}}}
\newcommand{\ConstantTok}[1]{\textcolor[rgb]{0.00,0.00,0.00}{#1}}
\newcommand{\ControlFlowTok}[1]{\textcolor[rgb]{0.13,0.29,0.53}{\textbf{#1}}}
\newcommand{\DataTypeTok}[1]{\textcolor[rgb]{0.13,0.29,0.53}{#1}}
\newcommand{\DecValTok}[1]{\textcolor[rgb]{0.00,0.00,0.81}{#1}}
\newcommand{\DocumentationTok}[1]{\textcolor[rgb]{0.56,0.35,0.01}{\textbf{\textit{#1}}}}
\newcommand{\ErrorTok}[1]{\textcolor[rgb]{0.64,0.00,0.00}{\textbf{#1}}}
\newcommand{\ExtensionTok}[1]{#1}
\newcommand{\FloatTok}[1]{\textcolor[rgb]{0.00,0.00,0.81}{#1}}
\newcommand{\FunctionTok}[1]{\textcolor[rgb]{0.00,0.00,0.00}{#1}}
\newcommand{\ImportTok}[1]{#1}
\newcommand{\InformationTok}[1]{\textcolor[rgb]{0.56,0.35,0.01}{\textbf{\textit{#1}}}}
\newcommand{\KeywordTok}[1]{\textcolor[rgb]{0.13,0.29,0.53}{\textbf{#1}}}
\newcommand{\NormalTok}[1]{#1}
\newcommand{\OperatorTok}[1]{\textcolor[rgb]{0.81,0.36,0.00}{\textbf{#1}}}
\newcommand{\OtherTok}[1]{\textcolor[rgb]{0.56,0.35,0.01}{#1}}
\newcommand{\PreprocessorTok}[1]{\textcolor[rgb]{0.56,0.35,0.01}{\textit{#1}}}
\newcommand{\RegionMarkerTok}[1]{#1}
\newcommand{\SpecialCharTok}[1]{\textcolor[rgb]{0.00,0.00,0.00}{#1}}
\newcommand{\SpecialStringTok}[1]{\textcolor[rgb]{0.31,0.60,0.02}{#1}}
\newcommand{\StringTok}[1]{\textcolor[rgb]{0.31,0.60,0.02}{#1}}
\newcommand{\VariableTok}[1]{\textcolor[rgb]{0.00,0.00,0.00}{#1}}
\newcommand{\VerbatimStringTok}[1]{\textcolor[rgb]{0.31,0.60,0.02}{#1}}
\newcommand{\WarningTok}[1]{\textcolor[rgb]{0.56,0.35,0.01}{\textbf{\textit{#1}}}}
\usepackage{graphicx,grffile}
\makeatletter
\def\maxwidth{\ifdim\Gin@nat@width>\linewidth\linewidth\else\Gin@nat@width\fi}
\def\maxheight{\ifdim\Gin@nat@height>\textheight\textheight\else\Gin@nat@height\fi}
\makeatother
% Scale images if necessary, so that they will not overflow the page
% margins by default, and it is still possible to overwrite the defaults
% using explicit options in \includegraphics[width, height, ...]{}
\setkeys{Gin}{width=\maxwidth,height=\maxheight,keepaspectratio}
\IfFileExists{parskip.sty}{%
\usepackage{parskip}
}{% else
\setlength{\parindent}{0pt}
\setlength{\parskip}{6pt plus 2pt minus 1pt}
}
\setlength{\emergencystretch}{3em}  % prevent overfull lines
\providecommand{\tightlist}{%
  \setlength{\itemsep}{0pt}\setlength{\parskip}{0pt}}
\setcounter{secnumdepth}{0}
% Redefines (sub)paragraphs to behave more like sections
\ifx\paragraph\undefined\else
\let\oldparagraph\paragraph
\renewcommand{\paragraph}[1]{\oldparagraph{#1}\mbox{}}
\fi
\ifx\subparagraph\undefined\else
\let\oldsubparagraph\subparagraph
\renewcommand{\subparagraph}[1]{\oldsubparagraph{#1}\mbox{}}
\fi

%%% Use protect on footnotes to avoid problems with footnotes in titles
\let\rmarkdownfootnote\footnote%
\def\footnote{\protect\rmarkdownfootnote}

%%% Change title format to be more compact
\usepackage{titling}

% Create subtitle command for use in maketitle
\newcommand{\subtitle}[1]{
  \posttitle{
    \begin{center}\large#1\end{center}
    }
}

\setlength{\droptitle}{-2em}

  \title{ST501 Group Project}
    \pretitle{\vspace{\droptitle}\centering\huge}
  \posttitle{\par}
    \author{Our Group}
    \preauthor{\centering\large\emph}
  \postauthor{\par}
      \predate{\centering\large\emph}
  \postdate{\par}
    \date{7/20/2018}


\begin{document}
\maketitle

\hypertarget{introduction}{%
\section{Introduction}\label{introduction}}

\hypertarget{part-1-convergences}{%
\section{Part 1 Convergences}\label{part-1-convergences}}

\hypertarget{a}{%
\subsection{A}\label{a}}

\[f_Y(y) = \frac{1}{2b}e^{-(\frac{|y-\mu|}{b})}\]

Given: \(\mu = 0\) and \(b=5\)

\[E(Y) = \int_{-\infty}^{\infty} y * f(y)dy\]
\[= \int_{-\infty}^0y*\frac{1}{10}e^{\frac{y}{5}} + \int^{\infty}_0y*\frac{1}{10}e^{\frac{-y}{5}}\]

\[= \frac{1}{10}\Bigg[\int_{-\infty}^0ye^{\frac{y}{5}} + \int^{\infty}_0y*e^{\frac{-y}{5}}\Bigg]\]

\[= \frac{1}{10}\Bigg[\Big[(5y-25)e^{\frac{y}{5}}\Big]_{-\infty}^0 + \Big[(-5y-25)e^{\frac{-y}{5}}\Big]_0^\infty\Bigg]\]
\[=\frac{1}{10}\Big[-25+25\Big]\] \[E(Y) = 0 \text{   }\blacksquare\]
\[E(Y^2) = \int_{-\infty}^{\infty} y^2 * f(y)dy\]

\[= \frac{1}{10}\Bigg[\int_{-\infty}^0y^2e^{\frac{y}{5}} + \int^{\infty}_0y^2*e^{\frac{-y}{5}}\Bigg]\]
\[= \frac{1}{10}*\Big[250+250\Big]\]

\[E(Y^2)= 50 \ \ \blacksquare\]

Therefor \(E(Y^2)\) exists.

Thus:

\(E(Y) = 0\), \(E(Y^2) = 50\) then \(Var(Y) = 50\) by the variance
computing formula.

\[L = \frac{1}{n}\sum_{i=1}^{n}Y_i^2\] By the Law of Large Numbers:

\[\frac{1}{n}\sum_{i=1}^{n}Y_i\ \ \underrightarrow{p} \ E(Y_i)\] By
generalisation of the Law of Large Numbers:

\[L = \frac{1}{n}\sum_{i=1}^{n}Y_i^2 \ \ \underrightarrow{p} \ E(Y_i^2) = 50\]
Therefore:

\[L \ \ \underrightarrow{p} \ \ 50 \ \ \blacksquare\]

\hypertarget{b}{%
\subsection{B}\label{b}}

By the continuity theorem,
\(K = \sqrt{L} \ \ \underrightarrow{p} \ \sqrt{50}\)

Thus: \(K\ \ \underrightarrow{p} \ \sqrt{50}\).

\hypertarget{c}{%
\subsection{C}\label{c}}

Derrive the CDF of the double exponential distribution.

\[f_Y(y) = \frac{1}{2b} * e^{-(\frac{|{y-\mu|)}}{b})}\]

This yields two cases for the CDF:

\[
F_Y(y)=
\begin{cases}
\int_{-\infty}^tf_Y(Y)dt & \text{for } y \le \mu \\
\int^{\infty}_tf_Y(Y)dt & \text{for } y \ge \mu \\
\end{cases}
\]

Solving for these leaves us with the following:

For \(y \le \mu\) :

\[\int_{-\infty}^yf_Y(Y)dt\]
\[\int_{-\infty}^y\frac{1}{2b} * e^{-(\frac{{t-\mu}}{b})}dt\] We know
that in this case we are solving for when the exponent is positive.

\[\int_{-\infty}^y\frac{1}{2b} * e^{(\frac{|{t-\mu|)}}{b})}dt\]

\[\frac{1}{2b} * e^{(\frac{t-\mu}{b})}*b|_{-\infty}^{y}=\frac{1}{2}e^{(\frac{t-\mu}{b})} - 0 = \frac{1}{2}e^{(\frac{t-\mu}{b})}\]

For the case where \(y \ge \mu\)

\[\int_{-\infty}^yf_Y(Y)dt\]

\[\int_{-\infty}^y\frac{1}{2b} * e^{-(\frac{{t-\mu}}{b})}dt\]

This must be further split into:

\[\int_{-\infty}^\mu\frac{1}{2b} * e^{-(\frac{{t-\mu}}{b})}dt + \int_{\mu}^y\frac{1}{2b} * e^{-(\frac{{t-\mu}}{b})}dt\]

\[\frac{1}{2b} * e^{(\frac{-t+\mu}{b})}*-b|_{-\infty}^{\mu} + \frac{1}{2b} * e^{(\frac{-t+\mu}{b})}*-b|_{\mu}^{y}\]

\[\frac{1}{2} - \frac{1}{2}e^{\frac{\mu-y}{b}} + \frac{1}{2}\]

Thus for this case:

\[1-\frac{1}{2}e^{\frac{\mu-y}{b}}\]

In conclusion the CDF for the double exponential distribution is:

\[F_Y(y)=
\begin{cases}
\frac{1}{2}e^{(\frac{y-\mu}{b})} & \text{for } y \le \mu \\
1-\frac{1}{2}e^{\frac{\mu-y}{b}} & \text{for } y \ge \mu
\end{cases}\]

\hypertarget{d}{%
\subsection{D}\label{d}}

First to generate these random values we need to create the inverse
functions:

For the case when: \[x\lt\mu\]

\[u = \frac{1}{2}e^{\frac{y-\mu}{b}}\]

\[b*log(2u) = x - \mu\]

Thus \[x = \mu + b*log(2u)\]

for \[u>0\]

\[x\lt\mu\]

For the case when

\[x \ge \mu\]

\[u = 1 - \frac{1}{2}e^{-\frac{y-\mu}{b}}\] \[-b*log(2-2u) = y- \mu\]

Thus \[\mu-b*log(2-2u) = y\]

For u\textless{}1, x \textgreater{}= mu

Using these inverse functions we can now do simulations.

\begin{Shaded}
\begin{Highlighting}[]
\KeywordTok{set.seed}\NormalTok{(}\DecValTok{336}\NormalTok{)}

\NormalTok{test_func<-}\ControlFlowTok{function}\NormalTok{(u)\{}
\NormalTok{  out_come_}\DecValTok{1}\NormalTok{ <-}\StringTok{ }\DecValTok{0} \OperatorTok{+}\StringTok{ }\DecValTok{5}\OperatorTok{*}\KeywordTok{log}\NormalTok{(}\DecValTok{2}\OperatorTok{*}\NormalTok{u)}
\NormalTok{  out_come_}\DecValTok{2}\NormalTok{ <-}\StringTok{ }\DecValTok{0} \OperatorTok{-}\StringTok{ }\DecValTok{5}\OperatorTok{*}\KeywordTok{log}\NormalTok{(}\DecValTok{2-2}\OperatorTok{*}\NormalTok{u)}
  \KeywordTok{cbind}\NormalTok{(out_come_}\DecValTok{1}\OperatorTok{^}\DecValTok{2}\NormalTok{, out_come_}\DecValTok{2}\OperatorTok{^}\DecValTok{2}\NormalTok{)}
\NormalTok{\}}

\NormalTok{a<-(}\KeywordTok{test_func}\NormalTok{(}\KeywordTok{seq}\NormalTok{(}\FloatTok{0.01}\NormalTok{,.}\DecValTok{999}\NormalTok{,.}\DecValTok{001}\NormalTok{)))}

\NormalTok{rdoublex <-}\StringTok{ }\ControlFlowTok{function}\NormalTok{(u, }\DataTypeTok{mu =} \DecValTok{0}\NormalTok{, }\DataTypeTok{b =} \DecValTok{5}\NormalTok{)\{}
  \ControlFlowTok{if}\NormalTok{( u }\OperatorTok{<}\StringTok{ }\FloatTok{0.5}\NormalTok{)\{}
\NormalTok{    out_come_}\DecValTok{1}\NormalTok{ <-}\StringTok{ }\NormalTok{mu }\OperatorTok{+}\StringTok{ }\NormalTok{b}\OperatorTok{*}\KeywordTok{log}\NormalTok{(}\DecValTok{2}\OperatorTok{*}\NormalTok{u)}
\NormalTok{    out_come_}\DecValTok{1}
\NormalTok{  \} }\ControlFlowTok{else}\NormalTok{\{}
\NormalTok{     out_come_}\DecValTok{2}\NormalTok{ <-}\StringTok{ }\NormalTok{mu }\OperatorTok{-}\StringTok{ }\NormalTok{b}\OperatorTok{*}\KeywordTok{log}\NormalTok{(}\DecValTok{2-2}\OperatorTok{*}\NormalTok{u)}
\NormalTok{     out_come_}\DecValTok{2}
\NormalTok{  \}}
    
\NormalTok{\}}


\NormalTok{output_matrix <-}\StringTok{ }\KeywordTok{matrix}\NormalTok{(}\DecValTok{0}\NormalTok{, }\DataTypeTok{nrow=} \DecValTok{250}\NormalTok{, }\DataTypeTok{ncol =} \DecValTok{50}\NormalTok{)}

\CommentTok{# Basic Function}
\ControlFlowTok{for}\NormalTok{( j }\ControlFlowTok{in} \DecValTok{1}\OperatorTok{:}\DecValTok{50}\NormalTok{)\{}
  \ControlFlowTok{for}\NormalTok{(i }\ControlFlowTok{in} \DecValTok{1}\OperatorTok{:}\DecValTok{250}\NormalTok{)\{}
\NormalTok{    random_value <-}\StringTok{ }\KeywordTok{runif}\NormalTok{(i, }\DecValTok{0}\NormalTok{, }\DecValTok{1}\NormalTok{)}
\NormalTok{    output_matrix[i, j]<-}\StringTok{ }\KeywordTok{mean}\NormalTok{(}\KeywordTok{vapply}\NormalTok{(random_value, }\DataTypeTok{FUN =}\NormalTok{ rdoublex, }\KeywordTok{double}\NormalTok{(}\DecValTok{1}\NormalTok{)))}
\NormalTok{  \}}
\NormalTok{\}}

\NormalTok{p1 <-}\StringTok{ }\NormalTok{output_matrix }\OperatorTok\StringTok{ }
\StringTok{  }\KeywordTok{as.data.frame}\NormalTok{() }\OperatorTok\StringTok{ }
\StringTok{  }\KeywordTok{mutate}\NormalTok{(}\DataTypeTok{N =} \DecValTok{1}\OperatorTok{:}\DecValTok{250}\NormalTok{) }\OperatorTok\StringTok{ }
\StringTok{  }\KeywordTok{gather}\NormalTok{(replication, value, }\OperatorTok{-}\NormalTok{N) }\OperatorTok\StringTok{ }
\StringTok{  }\KeywordTok{ggplot}\NormalTok{(}\KeywordTok{aes}\NormalTok{(N, value))}\OperatorTok{+}
\StringTok{  }\KeywordTok{geom_point}\NormalTok{(}\DataTypeTok{alpha=}\DecValTok{1}\OperatorTok{/}\DecValTok{5}\NormalTok{)}\OperatorTok{+}
\StringTok{  }\KeywordTok{theme_minimal}\NormalTok{()}\OperatorTok{+}
\StringTok{  }\KeywordTok{labs}\NormalTok{(}
    \DataTypeTok{title =} \StringTok{"Convergence of a Laplace Distribution with mu = 0, b = 5"}\NormalTok{,}
    \DataTypeTok{subtitle =} \StringTok{"50 Samples Drawn per Sample Size, N"}\NormalTok{,}
    \DataTypeTok{x =} \StringTok{"Sample Size, N"}
\NormalTok{  )}

\CommentTok{#L}

\NormalTok{output_matrix <-}\StringTok{ }\KeywordTok{matrix}\NormalTok{(}\DecValTok{0}\NormalTok{, }\DataTypeTok{nrow=} \DecValTok{250}\NormalTok{, }\DataTypeTok{ncol =} \DecValTok{50}\NormalTok{)}

\CommentTok{# Basic Function}
\ControlFlowTok{for}\NormalTok{( j }\ControlFlowTok{in} \DecValTok{1}\OperatorTok{:}\DecValTok{50}\NormalTok{)\{}
  \ControlFlowTok{for}\NormalTok{(i }\ControlFlowTok{in} \DecValTok{1}\OperatorTok{:}\DecValTok{250}\NormalTok{)\{}
\NormalTok{    random_value <-}\StringTok{ }\KeywordTok{runif}\NormalTok{(i, }\DecValTok{0}\NormalTok{, }\DecValTok{1}\NormalTok{)}
\NormalTok{    output_matrix[i, j]<-}\StringTok{ }\KeywordTok{mean}\NormalTok{(}\KeywordTok{vapply}\NormalTok{(random_value, }\DataTypeTok{FUN =}\NormalTok{ rdoublex, }\KeywordTok{double}\NormalTok{(}\DecValTok{1}\NormalTok{))}\OperatorTok{^}\DecValTok{2}\NormalTok{)}
\NormalTok{  \}}
\NormalTok{\}}

\NormalTok{limit_one <-}\StringTok{ }\DecValTok{20}
\NormalTok{p2 <-}\StringTok{ }\NormalTok{output_matrix }\OperatorTok\StringTok{ }
\StringTok{  }\KeywordTok{as.data.frame}\NormalTok{() }\OperatorTok\StringTok{ }
\StringTok{  }\KeywordTok{mutate}\NormalTok{(}\DataTypeTok{N =} \DecValTok{1}\OperatorTok{:}\DecValTok{250}\NormalTok{) }\OperatorTok\StringTok{ }
\StringTok{  }\KeywordTok{gather}\NormalTok{(replication, value, }\OperatorTok{-}\NormalTok{N) }\OperatorTok\StringTok{ }
\StringTok{  }\KeywordTok{ggplot}\NormalTok{(}\KeywordTok{aes}\NormalTok{(N, value))}\OperatorTok{+}
\StringTok{  }\KeywordTok{geom_point}\NormalTok{(}\DataTypeTok{alpha=}\DecValTok{1}\OperatorTok{/}\DecValTok{5}\NormalTok{)}\OperatorTok{+}
\StringTok{  }\KeywordTok{theme_minimal}\NormalTok{()}\OperatorTok{+}
\StringTok{  }\KeywordTok{labs}\NormalTok{(}
    \DataTypeTok{title =} \StringTok{"Convergence of a Laplace Distribution of Y_i^\{2\} with ~mu~= 0, b = 5"}\NormalTok{,}
    \DataTypeTok{subtitle =} \StringTok{"50 Samples Drawn per Sample Size, N"}\NormalTok{,}
    \DataTypeTok{x =} \StringTok{"Sample Size, N"}
\NormalTok{  )}\OperatorTok{+}
\StringTok{  }\KeywordTok{geom_hline}\NormalTok{(}\DataTypeTok{yintercept =} \DecValTok{50}\NormalTok{, }\DataTypeTok{size =} \DecValTok{1}\NormalTok{, }\DataTypeTok{color =} \StringTok{"blue"}\NormalTok{)}\OperatorTok{+}
\StringTok{  }\KeywordTok{geom_hline}\NormalTok{(}\DataTypeTok{yintercept =} \DecValTok{50} \OperatorTok{+}\StringTok{ }\NormalTok{limit_one, }\DataTypeTok{linetype =} \StringTok{"dotted"}\NormalTok{, }\DataTypeTok{color =} \StringTok{"red"}\NormalTok{, }\DataTypeTok{size =} \DecValTok{1}\NormalTok{)}\OperatorTok{+}
\StringTok{  }\KeywordTok{geom_hline}\NormalTok{(}\DataTypeTok{yintercept =} \DecValTok{50} \OperatorTok{-}\StringTok{ }\NormalTok{limit_one, }\DataTypeTok{linetype =} \StringTok{"dotted"}\NormalTok{, }\DataTypeTok{color =} \StringTok{"red"}\NormalTok{, }\DataTypeTok{size =} \DecValTok{1}\NormalTok{)}
\NormalTok{p2}
\end{Highlighting}
\end{Shaded}

\includegraphics{2018-07-27_ST501_Group_Project_files/figure-latex/unnamed-chunk-1-1.pdf}

\begin{Shaded}
\begin{Highlighting}[]
\CommentTok{#K}

\NormalTok{output_matrix <-}\StringTok{ }\KeywordTok{matrix}\NormalTok{(}\DecValTok{0}\NormalTok{, }\DataTypeTok{nrow=} \DecValTok{250}\NormalTok{, }\DataTypeTok{ncol =} \DecValTok{50}\NormalTok{)}

\CommentTok{# Basic Function}
\ControlFlowTok{for}\NormalTok{( j }\ControlFlowTok{in} \DecValTok{1}\OperatorTok{:}\DecValTok{50}\NormalTok{)\{}
  \ControlFlowTok{for}\NormalTok{(i }\ControlFlowTok{in} \DecValTok{1}\OperatorTok{:}\DecValTok{250}\NormalTok{)\{}
\NormalTok{    random_value <-}\StringTok{ }\KeywordTok{runif}\NormalTok{(i, }\DecValTok{0}\NormalTok{, }\DecValTok{1}\NormalTok{)}
\NormalTok{    output_matrix[i, j]<-}\StringTok{ }\KeywordTok{sqrt}\NormalTok{(}\KeywordTok{mean}\NormalTok{(}\KeywordTok{vapply}\NormalTok{(random_value, }\DataTypeTok{FUN =}\NormalTok{ rdoublex, }\KeywordTok{double}\NormalTok{(}\DecValTok{1}\NormalTok{))}\OperatorTok{^}\DecValTok{2}\NormalTok{))}
\NormalTok{  \}}
\NormalTok{\}}

\NormalTok{limit_two <-}\StringTok{ }\DecValTok{3}
\NormalTok{p3 <-}\StringTok{ }\NormalTok{output_matrix }\OperatorTok\StringTok{ }
\StringTok{  }\KeywordTok{as.data.frame}\NormalTok{() }\OperatorTok\StringTok{ }
\StringTok{  }\KeywordTok{mutate}\NormalTok{(}\DataTypeTok{N =} \DecValTok{1}\OperatorTok{:}\DecValTok{250}\NormalTok{) }\OperatorTok\StringTok{ }
\StringTok{  }\KeywordTok{gather}\NormalTok{(replication, value, }\OperatorTok{-}\NormalTok{N) }\OperatorTok\StringTok{ }
\StringTok{  }\KeywordTok{ggplot}\NormalTok{(}\KeywordTok{aes}\NormalTok{(N, value))}\OperatorTok{+}
\StringTok{  }\KeywordTok{geom_point}\NormalTok{(}\DataTypeTok{alpha=}\DecValTok{1}\OperatorTok{/}\DecValTok{5}\NormalTok{)}\OperatorTok{+}
\StringTok{  }\KeywordTok{theme_minimal}\NormalTok{()}\OperatorTok{+}
\StringTok{  }\KeywordTok{labs}\NormalTok{(}
    \DataTypeTok{title =} \StringTok{"Convergence of a Laplace Distribution of sqrt(Y_i^\{2\}) with mu= 0, b = 5"}\NormalTok{,}
    \DataTypeTok{subtitle =} \StringTok{"50 Samples Drawn per Sample Size, N"}\NormalTok{,}
    \DataTypeTok{x =} \StringTok{"Sample Size, N"}
\NormalTok{  )}\OperatorTok{+}
\StringTok{  }\KeywordTok{geom_hline}\NormalTok{(}\DataTypeTok{yintercept =} \FloatTok{7.07}\NormalTok{, }\DataTypeTok{size =} \DecValTok{1}\NormalTok{, }\DataTypeTok{color =} \StringTok{"blue"}\NormalTok{)}\OperatorTok{+}
\StringTok{  }\KeywordTok{geom_hline}\NormalTok{(}\DataTypeTok{yintercept =} \FloatTok{7.07} \OperatorTok{+}\StringTok{ }\NormalTok{limit_two, }\DataTypeTok{linetype =} \StringTok{"dotted"}\NormalTok{, }\DataTypeTok{color =} \StringTok{"red"}\NormalTok{, }\DataTypeTok{size =} \DecValTok{1}\NormalTok{)}\OperatorTok{+}
\StringTok{  }\KeywordTok{geom_hline}\NormalTok{(}\DataTypeTok{yintercept =} \FloatTok{7.07} \OperatorTok{-}\StringTok{ }\NormalTok{limit_two, }\DataTypeTok{linetype =} \StringTok{"dotted"}\NormalTok{, }\DataTypeTok{color =} \StringTok{"red"}\NormalTok{, }\DataTypeTok{size =} \DecValTok{1}\NormalTok{)}
\NormalTok{p3}
\end{Highlighting}
\end{Shaded}

\includegraphics{2018-07-27_ST501_Group_Project_files/figure-latex/unnamed-chunk-1-2.pdf}

\hypertarget{e}{%
\subsection{E}\label{e}}


\end{document}
